Z639: Social Media Mining								
Spring 2018	Group 2 - Project Proposal 								
Due Feb. 5

Elizabeth Supinski, Nandita Sathe, Sean Shiverick

Final project should address the following points:
Note: Final project must analyze Twitter data using Python.
Research proposal should have section headings, complete paragraphs 
and correct English grammar. 
Proposal should include the sections listed below:

I. Introduction (1-2 paragraphs) addresses the following:
What your main idea is and why it is interesting?
Clearly identify your topic; define key terms – cite as necessary
Motivate the study; tell the reader why it is important practically 
and theoretically
Identify a problem, e.g., lack of previous research and explain 
how your research addresses the problem
Ask a research question that follows from the above

II. Literature Review (2-4 paragraphs) and includes:
Establish the field you are working in - if little research has 
been done on your topic, find what comes closest
Cite at least 6 ACADEMIC, PEER REVIEWED RESEARCH PAPERS and at 
least one paper for each topic area your work falls into, including 
methodology

For methodological studies, explain the method that was proposed and 
how it was tested

For empirical studies, indicate what kind of data was analyzed, what 
research question was asked, and what was found

For conceptual studies, indicate what was claimed, on what basis
Show that a gap exists for your work
E.g., This is known about A, and that is known about B, but little is 
known about the relationship of A to B.  (This is the gap your study will start to fill)
Clearly cite the related works you discuss using an accepted reference method

III. Dataset Description should be 2-4 paragraphs and include:
Describe your publically available Twitter data subset.
Describe how the data is subset and justify your decisions. 
Your data may be temporally subset, meaning you collect "tweets" 
from a period of time. Your data may be subset by certain feature, 
meaning you only collect data from certain set of users, hashtags, categories. etc.

IV. Research Design and Methods (2-4 paragraphs):

Describe the main inference or analysis you would like to make with this data. 
This is the "end goal" of the project. At this stage, you don't have to know how 
you're going to get there, just what your main goal will be. 

For Robinson's Trump tweet analysis, the main goal was to identify 
Tweets written by Trump. 
For Culotta's flu paper, the main goal was to see if tweets about the 
flu and flu outbreaks were correlated. 
What do you want to do with the data?

Indicate what features you will have to extract from the data and what transformations you will have to do to reach your analytical goal. It's possible, but unlikely, that you can just use the raw data from the social media site as input to your analysis, but you will more likely have to transform the data for it to be usable. Will you use word counts? Do the data come with built-in features (author, time of posting, hashtags) that will be useful? Will you analyze your data as a network?

VI. Conclusions (2-3 paragraphs):
Reiterate importance or significance of your project and provide a brief summary of entire study. Answer:
Why should your study be done?
Why are you using the data and research design you have chosen over other options,
What are some potential real world, theoretical or methodological implications of your study?
How does your study fit within the broader scholarship about the research problem?

VI. References (Cite 6 academic sources, in APA or another accepted citation style)
