VI. References:

[1] National Institute on Drug Abuse (NIH). Opioid Overdose Crisis. National Institutes of Health (NIH), Washington D.C., Accessed online January, 2018. https://www.drugabuse.gov/drugs-abuse/opioids/opioid-overdose-crisis 

[2] Centers for Disease Control and Prevention (CDC). Opioid overdose: Understanding the epidemic.
Accessed online, January, 2018. https://www.cdc.gov/drugoverdose/index.html 

[3] Culotta, A. (2010). Towards detecting influenza epidemics by analyzing Twitter messages. 
Proceedings of the 1st Workshop on Social Media Analytics (SOMA 10), July 25, 2010. Washington, DC, USA. http://snap.stanford.edu/soma2010/papers/soma2010-16.pdf 

[4] Paul, M. J., Dredze, M., & Broniatowski, D. (2014). Twitter Improves Influenza Forecasting. PLoS Currents, 6, E-Currents.Outbreaks. http://doi.org/10.1371/currents.outbreaks.90b9ed0f59bae4ccaa683a39865d9117 
[5] Lazer et al., David Lazer, Ryan Kennedy, Gary King, and Alessandro Vespignani. (2014). The Parable of Google Flu: Traps in Big Data Analysis. Science, 343, 6176 (2014), 1203–1205. https://doi.org/10.1126/science.1248506 arXiv:http://science.sciencemag.org/content/343/6176/1203.full.pdf 

[6] Nascimento TD, DosSantos MF, Danciu T, DeBoer M, van Holsbeeck H, Lucas SR, et al. Real-time sharing and expression of migraine headache suffering on Twitter: a cross-sectional infodemiology study. J Med Internet Res. 2014;16(4):e96. https://www.ncbi.nlm.nih.gov/pmc/articles/PMC4004155/ 

[7] Chary, M., Genes, N., Giraud-Carrier, C., Hanson, C., Nelson, L. S., and Manini, A. F. Epidemiology from Tweets: Estimating Misuse of Prescription Opioids in the USA from Social Media. Journal of Medical Toxicolology. (2017) 13: 278. https://doi.org/10.1007/s13181-017-0625-5 

[8] Dzierak, L. (2017). AI Scans Twitter for Signs of Opioid Abuse. Scientific American. Published online, October 20, 2017 (retrieved January 21, 2018). https://www.scientificamerican.com/article/ai-scans-twitter-for-signs-of-opioid-abuse/ 

[9] Sarker, A., O’Connor, K., Ginn, R., Scotch, M., Smith, K., Malone, D., and Gonzalez, G. (2016).
Social Media Mining for Toxicovigilance: Automatic Monitoring of Prescription Medication Abuse from Twitter. Drug Safety, 39, 231-240. https://link.springer.com/article/10.1007/s40264-015-0379-4

[10] Mackey, T. K., Kalyanam, J., Katsuki, T., and Lanckriet G. Twitter-Based Detection of Illegal Online Sale of Prescription Opioid. AJPH (2017). 107(12), 1910-1915.Published Online: November 08, 2017.
http://ajph.aphapublications.org/doi/full/10.2105/AJPH.2017.303994 . 

[11] Hanson, C. L., Cannon, B., Burton, S., & Giraud-Carrier, C. (2013). An Exploration of Social Circles and Prescription Drug Abuse Through Twitter. Journal of Medical Internet Research, 15(9), e189. http://doi.org/10.2196/jmir.2741
https://www.ncbi.nlm.nih.gov/pmc/articles/PMC3785991/ 

[12] Vowles KE, McEntee ML, Julnes PS, Frohe T, Ney JP, van der Goes DN. Rates of opioid misuse, abuse, and addiction in chronic pain: a systematic review and data synthesis. Pain. 2015;156(4):569-576. doi:10.1097/01.j.pain.0000460357.01998.f1.
https://idhdp.com/media/400537/rates_of_opioid_misuse-_abuse-_and_addiction_in3.pdf 

[13] Carlson RG, Nahhas RW, Martins SS, Daniulaityte R. Predictors of transition to heroin use among initially non-opioid dependent illicit pharmaceutical opioid users: A natural history study. Drug Alcohol Depend. 2016;160:127-134. doi:10.1016/j.drugalcdep.2015.12.026. 
http://europepmc.org/articles/pmc4767561 

[14] Varshney, U. Pervasive healthcare and wireless health monitoring (2007). Mobile Networks and Applications, 12; 113-128, doi: 10.1007/s11036-007-0017-1
https://pdfs.semanticscholar.org/a1fa/da2df7d8a04abb5a6e5a271363b30ddfc09a.pdf 

[15] Garland, E.L., Bryan, C.J., Nakamura, Y., Froeliger, B., and Howard, M. O. Deficits in autonomic indices of emotion regulation and reward processing associated with prescription opioid use and misuse. Psychopharmacology (2017) 234: 621. https://doi.org/10.1007/s00213-016-4494-4 DOI: https://doi.org/10.1007/s0021 

[16] Eysenbach, G. (2009). Infodemiology and Infoveillance: Framework for an Emerging Set of Public Health Informatics Methods to Analyze Search, Communication and Publication Behavior on the Internet. Journal of Medical Internet Research, 11(1), e11. http://doi.org/10.2196/jmir.1157 

[17] Sarker, A., Ginn, R., Nikfarjam, A., O’Connor, K., Smith, K., Jayaraman, S., Upadhaya, T., and Gonzalez, G. Utilizing social media data for pharmacovigilance: A review, Journal of Biomedical Informatics, Volume 54, 2015, Pages 202-212, https://doi.org/10.1016/j.jbi.2015.02.004 
http://www.sciencedirect.com/science/article/pii/S1532046415000362 

[18] Dredze M, Paul MJ, Bergsma S, Tran H. Carmen: a twitter geolocation system with applications to public health. In AAAI workshop on expanding the boundaries of health informatics using AI. (HIAI). 2013 Jun 29 (pp. 20–24).


[19] Pastor-Satorras, R and Vespignani A. 2001. Epidemic spreading in scale-free networks. Phys. Rev. Leˆ., 86:3200–3203.

[20] Volkow, N. D., ‘Frieden, T. R., Hyde, P. S. and Cha, S. S. (2014). Medication-assisted therapies: Tackling the opioid-overdose epidemic. New England Journal of Medicine, 370(22):2063–2066, PMID: 24758595.

[21] Wa‹tts, D. J. and Strogatz, S .H. (1998). Collective dynamics of €small-world€ networks. Nature, 393(4): 440–442. http://leonidzhukov.net/hse/2014/socialnetworks/papers/watts-collective_dynamics-nature_1998.pdf 

[22] Travers, J. and Milgram, S. (1969). An experimental study of the small world problem. Sociometry, 32(4): 425-443. http://www.jstor.org/stable/pdf/2786545.pdf?acceptTC=true 

[23] Yavin Shaham, Uri Shalev, Lin Lu, Harriet de Wit, and Jane Stewart. ‘The reinstatement model of drug relapse: history, methodology and major €findings. Psychopharmacology, 168(1):3–20, Jul 2003.

[24] Granove‹er, M. S. (1973). ‘The strength of weak ties. American Journal of Sociology, 78(6):1360€“-1380. 
http://www.journals.uchicago.edu/doi/abs/10.1086/225469 

[25] Centola, D. (2010). The Spread of Behavior in an Online Social Network Experiment. Science, New Series, Vol. 329(5996) (3 September 2010), pp. 1194-1197
http://image.sciencenet.cn/olddata/kexue.com.cn/upload/blog/file/2010/12/2010122110322935115.pdf 

[26] Lord, S., Brevard, J., & Budman, S. (2011). Connecting to Young Adults: An Online Social Network Survey of Beliefs and Attitudes Associated With Prescription Opioid Misuse Among College Students. Substance Use & Misuse, 46(1), 66–76. http://doi.org/10.3109/10826084.2011.521371 

[27] Sarker, A. and Gonzalez, G. (2017). A corpus for mining drug-related knowledge from Twitter chatter: Language models and their utilities, Data in Brief, Volume 10, 2017, Pages 122-131, ISSN 2352-3409, https://doi.org/10.1016/j.dib.2016.11.056 
http://www.sciencedirect.com/science/article/pii/S2352340916307168

[28] Shiverick, S. M. (2017). Using machine learning classification of opioid addiction for big data health analytics. In Use Cases in Big Data Software and Analytics,  vol. 3, edited by G. von Laszewski, pp. 272-282. Indiana University-Bloomington.
