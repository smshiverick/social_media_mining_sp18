\documentclass[sigconf]{acmart}

\usepackage{graphicx}
\usepackage{hyperref}
\usepackage{todonotes}

%\usepackage{endfloat}
%\renewcommand{\efloatseparator}{\mbox{}} % no new page between figures

\usepackage{booktabs} % For formal tables

\settopmatter{printacmref=false} % Removes citation information below abstract
\renewcommand\footnotetextcopyrightpermission[1]{} % removes footnote with conference information in first column
\pagestyle{plain} % removes running headers


\newcommand{\TODO}[1]{\todo[inline]{#1}}
\newcommand{\DONE}[1]{DONE: \todo[inline,color=green!30]{#1}}


\begin{document}
\title{Mapping the Social Network of Prescription Opioid Misuse on Twitter}


\author{Liz Supinski, Nandita Sathe, Sean M. Shiverick}
\affiliation{%
  \institution{Indiana University-Bloomington}
}
\email{smshiver@indiana.edu}

% The default list of authors is too long for headers}
\renewcommand{\shortauthors}{Supinski, Sathe, Shiverick}


%\begin{abstract}
%This paper provides a sample of a \LaTeX\ document which conforms,
%somewhat loosely, to the formatting guidelines for
%ACM SIG Proceedings.
%\end{abstract}

\keywords{Big Data, Edge Computing i523}


\maketitle



\section{Introduction}



%%%%%%%%%%%%%%%%%%%%%%%%%%%%%%%%%%%%%%%%%%%%%%%%%%%%%%%%%%%%%%%%%%%%%%%%%%%%%%%
Over the past two decades, prescription opioid abuse and addiction in the U.S. has become a health crisis of epidemic proportion. Since 1999, the number of overdose deaths from prescription opioids such as oxycodone, hydrocodone, and methadone, has more than quadrupled [1,2]. Social media media mining (SMM) has been used to address public health concerns through “infodemiology” and “toxicovigilance”, which involves informatics for purposes of epidemiology, including strategies for monitoring patterns and prevalence of medication abuse [3, 4]. For example, researchers have tracked the outbreak of flu epidemics using internet searches and Twitter posts [5,6,7]. Twitter data has also been used to track the expression of migraine headaches [8], detect misuse and abuse of prescription opioids [4,9,10], and monitor illegal online sales of prescription opioids [11]. The relationship between Twitter posts and health outcomes is largely correlational, yet some research suggests that discussion of opioid misuse on Twitter may provide reinforcement for actual misuse of opioids [12]. Additional evidence would help to clarify the social network of Twitter users and the contexts of prescription opioid misuse. 

This project seeks to examine the relations among social media users posting content related to the misuse of prescription opioids (MUPO). Mapping social media content related to prescription opioid use may help detect individuals at risk for misusing prescription opioids, and identify factors related to opioid addiction. The main goal of this project is to map the social network of prescription opioid misuse on Twitter. This proposal outlines several steps toward accomplishing this goal, which include: (a) Textual analysis of Twitter posts related to prescription opioid misuse, (b) Visualization of the geospatial location of Twitter users who post content on MUPO, and (c) Analysis of the social network of Twitter users who tweet about the MUPO. 

Research Questions 
This project is motivated by the following questions: 
(i) Can opioid misusers be detected by their Twitter posts? 
Can [prescription] opioid misuse be detected from Twitter posts (e.g., #hashtags)? 
(ii) Where (geographically) are Twitter users discussing opioid misuse? 
Can we map the geographic location of [prescription] opioid misuse from Twitter posts? 
(iii) Can the social network of opioid misuse be mapped from Twitter users? 
Can the social network of [prescription] opioid misuse be mapped from Twitter? 

II. Literature Review 

The misuse and abuse of opioids in the U.S. is a crisis with serious public health consequences. In 2015, an estimated 2 million Americans suffered from substance use disorders related to prescription opioid pain relievers [1,2]. Of patients legitimately prescribed opioids for chronic pain, between 21% to 29% misused them, 8% to 12% developed an opioid use disorder, and approximately 4% to 6% transitioned to heroin [13,14]. Mobile health applications that monitor medication consumption have been developed to detect potential medication abuse [15].

Infodemiology, broadly, the analysis of internet-sourced health data for public health research, is still in its infancy. One of the best known infodemiological successes is the forecast of flu outbreaks by analyzing the correlation between the frequency of flu-related Twitter posts and official reports of influenza-like-illness from the CDC [5,6]. This research suggests that Twitter data can supplement official reports by providing information about the spread of flu 1 to 2 weeks earlier than the CDC-ILI. Because pharmaceutical manufacturers are obliged to monitor the safety of their drugs both before and after their release into the market, a field known as pharmacovigilance, the tools of infodemiology were quickly taken up in this field. Sarker et al [4] provide a methodological review studies using social media for pharmacovigilance, in which they point out that most studies rely on supervised learning with expert medical annotation of social media data. 

In the area of prescription opioid misuse (MUPO), several works stand out. Chary et al. [9] estimated misuse of prescription opioids from tweets, successfully incidence of tweets indicating opioid misuse with geographic incidence of opioid misuse as documented by the National Surveys on Drug Usage and Health (NSDUH). They used a customized version of Jiang-Conrath similarity [33] to determine the “kernal-weighted semantic distance” between tweets and then used k-means clustering to separate tweets in in opioid-misuse and non-opioid misuse categories. Sarker et al [17] came up with a supervised classification technique to distinguish posts suggesting MUPO from non-MUPO posts. Sarker and Gonzalez extended this work to present corpi for mining drug-related knowledge from Twitter using the word2vec tool to train shallow neural networks with two layers. [27] They have continued to enhance these models since publication, and made their trained models available to researchers at https://data.mendeley.com/datasets/dwr4xn8kcv/3. These models can be used for natural language processing using the gensim python library.

Correlating twitter behavior with public health data requires that the twitter data be tied to geographic locations. Twitter geolocation is disabled by default and only a small number of users choose to enable it. Leetaru et al. [34] estimate that about 2% of tweets include geographic metadata, and.Chary et al. [9] reported the same prevalence of GPS-tagged data. Chary et al. used Carmen [18] to geolocate tweets without geodata. Carmen is a library for geolocating tweets, available in both Java and Python implementations, which  Given a tweet, Carmen will return Location objects that represent a physical location. Carmen infers structured location information (country, state, county, city) using geocoordinates (if available) and and user profile information, but not tweet content. In an early paper on use of Twitter to explore health-related topics, Prier et al. [35] used the Twitter search API’s parameter to retrieve tweets on a state-by-state basis. Twitter geosearch method is proprietary, but the Twitter search API documentation [36] suggests a combination of device/gps coordinates, user provided profile location, and network/ip address are probably used to location to determine tweets that fit within the geocode search parameter.

Like other behaviors, opioid misuse may spread by social contact. In terms of contagious disease, spreading or diffusion occurs within contact networks of direct person-to-person interaction or other indirect pathways [19]. In the case of opioids, rather than describing persons as infected or uninfected, people may be considered as more or less susceptible to misuse, dependence, and addiction, depending on individual and environmental factors [20]. Furthermore, the structure of the contact network can inflƒuence epidemic spreading [21]. The “small-world” property of many networks describes how distant nodes in a network is reduced by the pattern of connections within the network. For example, any two random acquaintances in a social network are connected, on average, by five to six intermediaries [22]. Contact networks of drug use may have small-world properties as a few highly connected nodes can rapidly transmit contagion (i.e., drugs) throughout the network [23]. In the case of simple contagion, weak ties among acquaintances or infrequent associations can also provide shortcuts between distant nodes within the network [24], and facilitate the spread of contagion or drug use. Social media research suggests that discussion of prescription drug abuse on Twitter is potentially reinforcing of medication abuse within an online social network [12]. Experimental evidence has also shown that the structure of an artificially constructed online community influenced the spreading of behavior, as individuals who received reinforcement from multiple connections within their social network were more likely to adopt a behavior [25]. Behavior also spread farther and faster through with small-world networks that are clustered and latticed than random networks.

3. GAP in the literature 

Past research has analyzed the textual content of Twitter posts related to MUPO, the geographical location of Twitter users posting about MUPO, yet few studies have analyzed the structure of the social network of relations among Twitter users discussing MUPO. The present study addresses this gap in the literature by integrating different approaches to investigating MUPO in social media. Our proposal includes text analysis, geospatial mapping and network analysis to gain a better understanding of the structure of relations among Twitter users, to identify measures of centrality and community structure. 

III. Data Description (2-4 paragraphs)

Twitter provides a massive amount of information about individuals broadcasting their opinions, moods, and activities [31]. This information is spatial as well as temporal. This project will analyse tweets between year 2015 to 2016 having at least one of the following keywords. (NSDUH for 2017 data is not yet available, so we will restrict our Twitter search to the most recent period for which we have correlate data available.) Twitter API will be used to extract the dataset. Our keyword list is derived from Chary et al. [9],  Mackey et al. [11] and ten most common prescription opioids from NSDUH-2015 [30] lists of opioid-detection keywords. The target terms to investigate are: oxycodone, oxycontin; hydrocodone, vicodin; codeine, percodan; percocet; lortab; lorcet; morphine, oxymorphone, hydromorphone, demerol, dilaudid; fentanyl, methadone; lomotil; kadian; avinza; duragesic; tramadol, buprenorphine, propoxyphene, roxanol.

We will use the methodology described by Sarker et al [27] to enhance the list with common misspellings, and web resources to identify common street names for the most commonly misused opioids, aiming for a final keyword list of 20-15 terms. We will also use latitude and longitude information of users from Twitter metadata. The spatial information will be used to map the geographic locations where opioid misuse has been taking place. We will also extract the network structure of relations among Twitter users. 

%%%%%%%%%%%%%%%%%%%%%%%%%%%%%%%%%%%%%%%%%%%%%%%%%%%%%%%%%%


%\begin{figure}[!ht]
%  \centering\includegraphics[width=\columnwidth]{images/fly.pdf}
%  \caption{Example caption}\label{f:fly}
%\end{figure}

When copying the example, please do not check in the images from the
examples into your images directory as you will not need them for your
paper. Instead use images that you like to include. If you do not have
any images, do not dreate the images folder.

\section{Tables}

In case you need to create tables, you can do this with online tools
(if you do not mind sharing your data) such as
\url{https://www.tablesgenerator.com/} or other such tools (please
google for them). They even allow you to manage tables as CSV.

% or generate them by hand while using the provided template in Table\ref{t:mytable}. 
% Notebthatbthe caption is before the tabular environment.

%\begin{table}[htb]
%\centering
%\caption{My caption}
%\label{t:mytabble}
%\begin{tabular}{lll}
%1 & 2 & 3 \\
%\hline
%4 & 5 & 6 \\
%7 & 8 & 9
%\end{tabular}
%\end{table}

\section{Long example}

If you like to see a more elaborate example, please look at
report-long.tex. 

\section{Conclusion}

Put here an conclusion. Conlcusions and abstracts must not have any
citations in the section.


%\begin{acks}
%  The authors would like to thank Dr. Gregor von Laszewski for his
%  support and suggestions to write this paper.
%\end{acks}

\bibliographystyle{ACM-Reference-Format}
\bibliography{report} 

\appendix
We include an appendix with common issues that we see when students
submit papers. One particular important issue is not to use the
underscore in bibtex labels. Sharelatex allows this, but the
proceedings script we have does not allow this.

%\input{issues}

\end{document}
